\documentclass[preview]{standalone}
\usepackage[english]{babel}
\usepackage{amsmath}
\usepackage{amssymb}
\begin{document}
\begin{align*}
\documentclass[preview]{standalone}
                \usepackage{amsmath}  % 数学符号包
                \usepackage{amssymb}  % 更多数学符号
                \usepackage{enumitem} % 列表样式
                \usepackage[UTF8]{ctex}
                \usepackage{bm}
                \usepackage{amsthm}
                \everymath{\displaystyle}  % 让所有数学模式都使用 \displaystyle
                \newcommand{\lb}{\left\llbracket}
                \newcommand{\rb}{\right\rrbracket}

                令 $A = \sum_{i = 1}^{n}a_i^p, B = \sum_{i = 1}^{n}b_i^q$ \\
                $p > 1$ 时,对 $\forall i \in \{1, 2, \cdots , n\}$, 由 \textbf{Young 不等式},
                \[
                    \frac{a_ib_i}{A^{\frac{1}{p}}B^{\frac{1}{q}}} \leqslant \frac{1}{p}\frac{a_i^p}{A} + \frac{1}{q}\frac{b_i^p}{B}
                \]
                故 
                \[
                    \frac{1}{A^{\frac{1}{p}}B^{\frac{1}{q}}}\sum_{i = 1}^{n}a_ib_i \leqslant \frac{1}{p} + \frac{1}{q} = 1
                \]
                即
                \[
                    \sum_{i = 1}^{n}a_ib_i \leqslant A^{\frac{1}{p}}B^{\frac{1}{q}} = \left(\sum_{i = 1}^{n}a_i^p\right)^{\frac{1}{p}} \cdot \left(\sum_{i = 1}^{n}b_i^q\right)^{\frac{1}{q}}
                \]
\end{align*}
\end{document}